\documentclass[12 pt]{article}
%\usepackage[utf8]{inputenc}
\usepackage{amsmath}
\usepackage{amssymb}
\usepackage{geometry}
\usepackage{fancyhdr}
\usepackage{setspace}
%\usepackage{mathptmx}
\usepackage{newtxtext,newtxmath}
%\usepackage{lineno}
\usepackage[english]{babel}

\renewcommand\headrulewidth{0 pt}
\pagestyle{fancy}
\fancyhf{}
\rfoot{\thepage}

\setstretch{2}

\geometry{tmargin=2.5 cm, bmargin=3 cm, lmargin=2.6 cm, rmargin=2.6 cm}

\begin{document}
\begin{Large}
\begin{center}
\textbf{Supplementary information}
\end{center}
\end{Large}
\begin{large}
\textbf{A temperature-driven model of phenological mismatch provides insights into the potential impacts of climate change on consumer-resource interactions} \\
\vspace{1 cm}
Portalier S.M.J.$^{1}$, Candau J.N.$^2$, Lutscher F.$^{1,3}$ \\
\end{large}
$^1$: Department of Mathematics and Statistics, University of Ottawa, Ottawa, ON, Canada \\
$^2$: Natural Resources Canada, Canadian Forest Service, Great Lakes Forestry Centre, Sault Ste. Marie, ON, Canada\\
$^3$: Department of Biology, University of Ottawa, Ottawa, ON, Canada \\
 
\vspace{1 cm}
In this supplementary material, we give the details for the mathematical derivation of the two sensitivity formulas for the end time of the seasonal resting period of a species. The general equation that connects the start time $t_0$, the rate curve $R(x)$ and the threshold $F$ to the end time $t^*$ of the resting period is

\stepcounter{equation}
\begin{equation}
    \int _{t_0} ^{t^*} R(x(t)) \mathrm{d}t = F. \tag*{Eq. S\theequation}
\end{equation}

\subsection*{General features}
We want to determine how $t^*$ changes when the temperature $x = x(t)$ changes by a small amount. More formally, we will derive a formula for the linear approximation

\stepcounter{equation}
\begin{equation}
    t^*(\epsilon) = t^*(0) + \epsilon \frac{d t^*}{d\epsilon} \tag*{Eq. S\theequation}
\end{equation}
where $\epsilon$ measures the magnitude of the small change, $t^*(0)$ is the end time when there is no change in the temperature time series from historical data, and the derivative is the sensitivity of the end time with respect to small changes. \par
We write the change in temperature as $x(t) + \epsilon z(t)$, where $z(t)$ is the pattern in which the temperature differs from the expectation and $\epsilon$ is small.  Since the end time now depends on $\epsilon$, we write $t^*=t^* (\epsilon)$.  The sensitivity of the end time with respect to $\epsilon$ is given by the derivative
\stepcounter{equation}
\begin{equation}
    \frac{\mathrm{d}t^*}{\mathrm{d}\epsilon} \; \text{for} \; \epsilon = 0. \tag*{Eq. S\theequation}
\end{equation}
This expression will depend on the pattern of temperature difference, $z(t)$. We will discuss two specific patterns below. \par

When we substitute these expressions into the defining equation for $t^*$ above, $\epsilon$ appears twice: once in the upper limit of integration and once in the integrand. To emphasize these two occurrences, we write the left-hand side of the equation as a function of two variables, namely
\stepcounter{equation}
\begin{equation} \label{Iequation}
    I(t^*(\epsilon),R(x+\epsilon x)) = \int _{t_0} ^{t^*(\epsilon)} R(x(t))+\epsilon z(t) \mathrm{d}t \tag*{Eq. S\theequation}
\end{equation}
When we differentiate the equation that defines the end time, $I(t^* ,R)=F$, with respect to $\epsilon$, we use the chain rule repeatedly and obtain
\stepcounter{equation}
\begin{equation} 
    \frac{\mathrm{d}}{\mathrm{d}\epsilon}I(t^*(\epsilon),R(x+\epsilon x)) = \frac{\partial I}{\partial t^*} \frac{\mathrm{d}t^*}{\mathrm{d}\epsilon}+\frac{\partial I}{\partial R} \frac{\mathrm{d} R}{\mathrm{d} x} \frac{\mathrm{d}x}{\mathrm{d}\epsilon} = 0 \tag*{Eq. S\theequation}
\end{equation}
The derivative of the integral in \ref{Iequation} with respect to the end time is simply the integrand evaluated at the end time. The derivative of the integral with respect to the integrand is the integral itself since this is linear. The derivative of the rate function with respect to $x$ is the usual derivative and the derivative of $x$ with respect to $\epsilon$ is $z$, by our definition above. Then we can solve the above equation for the quantity we are looking for and find
\stepcounter{equation}
\begin{equation} \label{dtdepsilon}
    \frac{\mathrm{d}t^*}{\mathrm{d}\epsilon}=\frac{- \int _{t_0} ^{t^*} R'(x(t)) z(t) \tag*{Eq. S\theequation} \mathrm{d}t}{R(x(t^*))}
\end{equation}
Hence, the end time has the linear approximation
\stepcounter{equation}
\begin{equation}
    t^*(\epsilon) \approx t^*(0)+\epsilon \frac{\mathrm{d}t^*}{\mathrm{d}\epsilon}=t^*(0)+\epsilon \frac{- \int _{t_0} ^{t^*} R'(x(t)) z(t) \mathrm{d}t}{R(x(t^*))} \tag*{Eq. S\theequation}
\end{equation}
As expected, the pattern by which the temperature deviates, $z(t)$, appears in this formula. We look at two interesting special cases for this pattern. \par

\subsection*{Specific patterns}
The first case is that the temperature change is constant throughout the period, independent of time. In that case, we can set $\epsilon z(t)=\Delta x$ to be the constant temperature difference. Then the function $z(t)$ drops out of the above integral and the end time is given by
\stepcounter{equation}
\begin{equation}
    t^*(\epsilon) \approx t^*(0) - \Delta x \frac{- \int _{t_0} ^{t^*} R'(x(t)) \mathrm{d}t}{R(x(t^*))} \tag*{Eq. S\theequation}
\end{equation}
Since $R'(x)>0$ and $R(x)>0$, the end time decreases if the temperature increases, i.e., the phenology advances. We knew this already from general consideration, but now we have an explicit expression for how much the advance is per degree increase. \par

The second case in which we can simplify the general formula is that there is a warm or cold spell of relatively short duration at a particular time during the resting phase. Then $\epsilon z(t)=\Delta x$ during the spell of duration $\Delta t$, starting at time $t_s$, and $z(t)=0$ otherwise. The integral in the numerator of \ref{dtdepsilon} can be approximated by
\stepcounter{equation}
\begin{equation}
    \epsilon \int _{t_0} ^{t^*} R'(x(t)) z(t) \mathrm{d}t = \Delta x \int _{t_s} ^{t_s + \Delta t}R'(x(t)) \mathrm{d}t \approx \Delta x \Delta t R'(x(t_s)) \tag*{Eq. S\theequation}
\end{equation}
Hence, the expression for the end time is approximately
\stepcounter{equation}
\begin{equation}
    t^*(\epsilon) \approx t^*(0)-\Delta x \frac{\Delta t R'(x(t_s))}{R(x(t^*))} \tag*{Eq. S\theequation}
\end{equation}
This means that the end time is most sensitive to a warm or cold spell when the derivative of the rate function is the highest, all other things being equal. \par

\subsection*{Derivative of the rate function}
\stepcounter{equation}
\begin{equation}
    R(x)=\frac{1}{1+exp(b(x-c))}, \tag*{Eq. S\theequation}
\end{equation}
we can explicitly calculate the derivative as
\stepcounter{equation}
\begin{equation}
    R'(x)=\frac{-b exp(b(x-c))}{(1+exp(b(x-c)))^2}, \tag*{Eq. S\theequation}
\end{equation}
which is positive since $b$ is negative. To find the maximum of the derivative, we differentiate again and find
\stepcounter{equation}
\begin{equation}
    R''(x) = \frac{-b^2 exp(b(x-c))(1-exp(b(x-c)))}{(1+exp(b(x-c)))^3} \tag*{Eq. S\theequation}
\end{equation}
The maximum of $R’$ occurs where $R’’ = 0$, which happens when $x = c$ (see Fig. 2B).
\end{document}