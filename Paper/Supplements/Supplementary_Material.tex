\documentclass[12 pt]{article}
%\usepackage[utf8]{inputenc}
\usepackage{amsmath}
\usepackage{amssymb}
\usepackage{geometry}
\usepackage{fancyhdr}
\usepackage{setspace}
%\usepackage{mathptmx}
\usepackage{newtxtext,newtxmath}
\usepackage{graphicx}
%\usepackage{lineno}
\usepackage[labelfont=bf]{caption}
\usepackage[english]{babel}
\usepackage[round]{natbib}

\renewcommand\headrulewidth{0 pt}
\pagestyle{fancy}
\fancyhf{}
%\rfoot{\thepage}
%\lfoot{A temperature-driven model of phenological mismatch \\ Portalier, Candau \& Lutscher}
%\lhead{\includegraphics[height=1 cm, keepaspectratio]{JAE_Logo}}

\setstretch{2}

\geometry{tmargin=2.5 cm, bmargin=3 cm, lmargin=2.6 cm, rmargin=2.6 cm}

\bibliographystyle{besjournals}

\begin{document}
\begin{Large}
\begin{center}
\textbf{Supplementary information}
\end{center}
\end{Large}
\begin{large}
\textbf{A temperature-driven model of phenological mismatch provides insights into the potential impacts of climate change on consumer-resource interactions} \\
\vspace{1 cm}
Portalier S.M.J.$^{1}$, Candau J.N.$^2$, Lutscher F.$^{1,3}$ \\
\end{large}
$^1$: Department of Mathematics and Statistics, University of Ottawa, Ottawa, ON, Canada \\
$^2$: Natural Resources Canada, Canadian Forest Service, Great Lakes Forestry Centre, Sault Ste. Marie, ON, Canada\\
$^3$: Department of Biology, University of Ottawa, Ottawa, ON, Canada \\
 
\vspace{0.5 cm}
\section{Theoretical developments}
In this supplementary material, we give the details for the mathematical derivation of the two sensitivity formulas for the end time of the seasonal resting period of a species. The general equation that connects the start time $t_0$, the rate curve $R(x)$ and the threshold $F$ to the end time $t^*$ of the resting period is

\stepcounter{equation}
\begin{equation}
    \int _{t_0} ^{t^*} R(x(t)) \mathrm{d}t = F. \tag*{Eq. S\theequation}
\end{equation}

\subsection*{General features}
We want to determine how $t^*$ changes when the temperature $x = x(t)$ changes by a small amount. More formally, we will derive a formula for the linear approximation

\stepcounter{equation}
\begin{equation}
    t^*(\epsilon) = t^*(0) + \epsilon \frac{d t^*}{d\epsilon} \tag*{Eq. S\theequation}
\end{equation}
where $\epsilon$ measures the magnitude of the small change, $t^*(0)$ is the end time when there is no change in the temperature time series from historical data, and the derivative is the sensitivity of the end time with respect to small changes. \par
We write the change in temperature as $x(t) + \epsilon z(t)$, where $z(t)$ is the pattern in which the temperature differs from the expectation and $\epsilon$ is small.  Since the end time now depends on $\epsilon$, we write $t^*=t^* (\epsilon)$.  The sensitivity of the end time with respect to $\epsilon$ is given by the derivative
\stepcounter{equation}
\begin{equation}
    \frac{\mathrm{d}t^*}{\mathrm{d}\epsilon} \; \text{for} \; \epsilon = 0. \tag*{Eq. S\theequation}
\end{equation}
This expression will depend on the pattern of temperature difference, $z(t)$. We will discuss two specific patterns below. \par

When we substitute these expressions into the defining equation for $t^*$ above, $\epsilon$ appears twice: once in the upper limit of integration and once in the integrand. To emphasize these two occurrences, we write the left-hand side of the equation as a function of two variables, namely
\stepcounter{equation}
\begin{equation} \label{Iequation}
    I(t^*(\epsilon),R(x+\epsilon x)) = \int _{t_0} ^{t^*(\epsilon)} R(x(t)+\epsilon z(t)) \mathrm{d}t. \tag*{Eq. S\theequation}
\end{equation}
When we differentiate the equation that defines the end time, $I(t^* ,R)=F$, with respect to $\epsilon$, we use the chain rule repeatedly and obtain
\stepcounter{equation}
\begin{equation} 
    \frac{\mathrm{d}}{\mathrm{d}\epsilon}I(t^*(\epsilon),R(x+\epsilon x)) = \frac{\partial I}{\partial t^*} \frac{\mathrm{d}t^*}{\mathrm{d}\epsilon}+\frac{\partial I}{\partial R} \frac{\mathrm{d} R}{\mathrm{d} x} \frac{\mathrm{d}x}{\mathrm{d}\epsilon} = 0. \tag*{Eq. S\theequation}
\end{equation}
The derivative of the integral in \ref{Iequation} with respect to the end time is simply the integrand evaluated at the end time. The derivative of the integral with respect to the integrand is the integral itself since this is linear. The derivative of the rate function with respect to $x$ is the usual derivative and the derivative of $x$ with respect to $\epsilon$ is $z$, by our definition above. Then we can solve the above equation for the quantity we are looking for and find
\stepcounter{equation}
\begin{equation} \label{dtdepsilon}
    \frac{\mathrm{d}t^*}{\mathrm{d}\epsilon}=\frac{- \int _{t_0} ^{t^*} R'(x(t)) z(t) \mathrm{d}t}{R(x(t^*))}. \tag*{Eq. S\theequation}
\end{equation}
Hence, the end time has the linear approximation
\stepcounter{equation}
\begin{equation}
    t^*(\epsilon) \approx t^*(0)+\epsilon \frac{\mathrm{d}t^*}{\mathrm{d}\epsilon}=t^*(0)+\epsilon \frac{- \int _{t_0} ^{t^*} R'(x(t)) z(t) \mathrm{d}t}{R(x(t^*))}. \tag*{Eq. S\theequation}
\end{equation}
As expected, the pattern by which the temperature deviates, $z(t)$, appears in this formula. We look at two interesting special cases for this pattern. \par

\subsection*{Specific patterns}
The first case is that the temperature change is constant throughout the period, independent of time. In that case, we can set $\epsilon z(t)=\Delta x$ to be the constant temperature difference. Then the function $z(t)$ drops out of the above integral and the end time is given by
\stepcounter{equation}
\begin{equation}\label{byconstant}
    t^*(\epsilon) \approx t^*(0) - \Delta x \frac{\int _{t_0} ^{t^*} R'(x(t)) \mathrm{d}t}{R(x(t^*))}. \tag*{Eq. S\theequation}
\end{equation}
Since $R'(x)>0$ and $R(x)>0$, the end time decreases if the temperature increases, i.e., the phenology advances. We knew this already from general consideration, but now we have an explicit expression for how much the advance is per degree increase. \par

The second case in which we can simplify the general formula is that there is a warm or cold spell of relatively short duration at a particular time during the resting phase. Then $\epsilon z(t)=\Delta x$ during the spell of duration $\Delta t$, starting at time $t_s$, and $z(t)=0$ otherwise. The integral in the numerator of \ref{dtdepsilon} can be approximated by
\stepcounter{equation}
\begin{equation}
    \epsilon \int _{t_0} ^{t^*} R'(x(t)) z(t) \mathrm{d}t = \Delta x \int _{t_s} ^{t_s + \Delta t}R'(x(t)) \mathrm{d}t \approx \Delta x \Delta t R'(x(t_s)) \tag*{Eq. S\theequation}
\end{equation}
Hence, the expression for the end time is approximately
\stepcounter{equation}
\begin{equation}\label{byspell}
    t^*(\epsilon) \approx t^*(0)-\Delta x \frac{\Delta t R'(x(t_s))}{R(x(t^*))} \tag*{Eq. S\theequation}
\end{equation}
This means that the end time is most sensitive to a warm or cold spell when the derivative of the rate function is the highest, all other things being equal. \par
The two formulas (\ref{byconstant} and \ref{byspell}) may seem different, but they express the same idea. One has to integrate $R'$ for all times where the two time series differ. When the two time series differ by a constant for all times, the one has to integrate over the entire time series. When the two time series differ only on an interval of length $\Delta t$, then one has to integrate over only that interval. If the interval is short, then the value of $R(x(t))$ does not change much and therefore the integral is approximated by the product of the length of the interval ($\Delta t$) and the value of the integrand ($R(x(t_s))$). 

\subsection*{Derivative of the rate function}
\stepcounter{equation}
\begin{equation}
    R(x)=\frac{1}{1+\mathrm{exp}(b(x-c))}, \tag*{Eq. S\theequation}
\end{equation}
we can explicitly calculate the derivative as
\stepcounter{equation}
\begin{equation}
    R'(x)=\frac{-b \mathrm{exp}(b(x-c))}{(1+\mathrm{exp}(b(x-c)))^2}, \tag*{Eq. S\theequation}
\end{equation}
which is positive since $b$ is negative. To find the maximum of the derivative, we differentiate again and find
\stepcounter{equation}
\begin{equation}
    R''(x) = \frac{-b^2 \mathrm{exp}(b(x-c))(1-\mathrm{exp}(b(x-c)))}{(1+\mathrm{exp}(b(x-c)))^3} \tag*{Eq. S\theequation}
\end{equation}
The maximum of $R’$ occurs where $R’’ = 0$, which happens when $x = c$ (see Fig. 2).

\subsection*{Illustration with a simplified time series}
In reality, the periods of high sensitivity of the two species may overlap more or less, and the rate functions at emergence time (the terms in the denominators in Eqs 3 and 4, main text) could differ significantly. As a result, the effect of temperature increases depends on details of each scenario. We illustrate this dependence using a simplified time series of daily mean temperatures as modelled by
\stepcounter{equation}
\begin{equation}
	x_i (t)=6.9+15 \; \text{cos} \left( \frac{2 \pi (t-200)}{365} \right)	\tag*{Eq. S\theequation}
\end{equation}
where the mean, amplitude and offset have been chosen to match historical averages in Fredericton (NB, Canada).  %We denote the end times of the consumer for the time series $x_i(t)$ by $t_{e,i}^*$ (emergence time) and of the resource by $t_{b,i}^*$ (budburst time).

%\clearpage
\begin{figure}[ht]
\begin{center}
\renewcommand{\thefigure}{S\arabic{figure}}
%\renewcommand\figurename{figure S}
\setcounter{figure}{0}
\includegraphics[width = 16 cm, keepaspectratio]{FigureS1}
\caption{\doublespacing Effects of a constant temperature difference on species phenology. Black is the consumer (SBW), and grey is the resource (balsam fir). A constant temperature difference advances species phenology. Dotted is the predicted value (Eq. 3 used with the $R$ functions of SBW and balsam fir), dashed is the linear approximation from the model with simple time series.}
\end{center}
\end{figure}

\clearpage
\section{Phenological models of balsam fir and spruce budworm}
\subsection{Phenological model of balsam fir's budburst}
We use the \textit{Uniforc} model of Chuine (2000) to model balsam fir's budburst phenology. \textit{Uniforc} predicts bud development as a function of temperature in the second stage of seasonal resting (i.e., ecodormancy). The heat accumulation rate writes
\stepcounter{equation}
\begin{equation}
   R(x)=\frac{1}{1+\text{exp}(b-(x-c))}.  \tag*{Eq. S\theequation}
\end{equation}
Accumulation starts some time after January 1st (Desbiens, 2007), when trees have accumulated enough cold to end bud dormancy. Budburst occurs when accumulation reaches a threshold $F^*$. 
\par
We fitted the \textit{Uniforc} model to budburst phenology data collected from 1980 to 1996 in Quebec and New Brunswick (Desbiens, 2007; R\'egni\`ere pers. comm., 2020) (see Fig. S2). Each year, bud development was observed in different sites during the growing season at time intervals ranging from two days to two weeks. Budburst occurs when buds develop from class I to II according to the class scheme developed by \cite{Dorais1982}. The budburst date was defined as the date when $50 \%$ of the buds in the site have reached stage II. We obtained temperature data at each site for each year using BioSIM. We estimated parameter values of the \textit{Uniforc} model using simulated annealing in order to predict budburst date according to temperatures during development period.
\par
Fitting the \textit{Uniforc} model to phenological data resulted in the parameter values: $b = -1.32$, $c = 7.14 \; ^\circ$C, $t_0 = 87$ (March 28th), and $F^* = 18.6$. 
To evaluate model's goodness of fit, we first computed the root mean squared error (RMSE), which is the standard deviation of the residuals. Then, we tested the slope and intercept of the observed versus predicted data of the fitted model. A slope that does not significantly differ from 1, and an intercept that does not significantly differ from 0 mean that the model is unbiased \citep{Pineiro2008}. Since the same dataset was used to fit and to test the model, we performed a leave-one-out cross validation. 
\par
We found RMSE $= 12.6$. The slope of the regression of the observed versus predicted data does not significantly differ from 1 ($p = 0.42$), and the intercept does not significantly differ from 0 ($p = 0.38$) (Fig. S3A). Thus, the model is considered unbiased. Moreover, the residuals of this fitting follow a Normal distribution centred on 0 (Fig. S3B). There is no obvious pattern for the residuals across latitude in the range of our study (Fig. S3C). 
\par

\subsection{Sensitivity analysis of the balsam fir and spruce budworm's models}
We performed sensitivity analysis on both models using partial rank correlation coefficients \citep{Wu2013}. The budworm model is sensitive to most parameters (Fig. S4A). The only exception is $x_m$ (the maximal temperature) since very high temperatures are rare during late winter and spring, and to a certain extent $\beta _1$. Increasing parameters $\beta _2$, $\beta _4$, and $x_b$ (minimal temperature) delays emergence, while increasing $\beta _3$ strongly advances phenology. The tree model is most sensitive to parameters $t_0$ (when the tree starts accumulating heat) and $b$ (which drives the speed of accumulation). An increase in $t_0$ postpones phenology, while an increase in $b$ advances it (Fig. S4B).

\begin{figure}[ht]
\begin{center}
\renewcommand{\thefigure}{S\arabic{figure}}
%\setcounter{figure}{0}
\includegraphics[width = 16 cm, keepaspectratio]{map_fitting}
\caption{\doublespacing Location of the sample sites where data on phenology of balsam fir's buds was collected.}
\end{center}
\end{figure}

\begin{figure}[ht]
\begin{center}
\renewcommand{\thefigure}{S\arabic{figure}}
%\setcounter{figure}{0}
\includegraphics[width = 16 cm, keepaspectratio]{Fitting_Plots}
\caption{\doublespacing Evaluation of the goodness of fit of the balsam fir model. (A) Regression of observed versus predicted data. The slope does not significantly differ from 1 and the intercept from 0 (black line). (B) Residuals follow a Normal distribution centered on 0. (C) No obvious latitudinal patterns can be found on the residuals within the range of latitudes that is used throughout the rest of the study.}
\end{center}
\end{figure}

\begin{figure}[ht]
\begin{center}
\renewcommand{\thefigure}{S\arabic{figure}}
%\setcounter{figure}{0}
\includegraphics[width = 14 cm, keepaspectratio]{Sensitivity}
\caption{\doublespacing Partial Rank Correlation Coefficient (PRCC) of the budworm and tree's models. (A) The SBW model is sensitive to most parameters especially $\beta _2$, $\beta _4$ and $x_b$ that delay emergence, and $\beta _3$ that advances phenology. (B) The tree model is mostly sensitive to $b_f$  that hastens budburst, and $t_0$ that delays budburst.}
\end{center}
\end{figure}

\clearpage

\section{Analysis of variance}
Here, we present  the full results of the analysis of variance done on emergence date, budburst date and mismatch across latitude, for past/present temperatures, and for the three RCP scenarios. There are $6$ sites ranged from site 1 (southern site: $44.5^{\circ}$ N) to site 6 (northern site: $49.5^{\circ}$ N) (see Fig. 4 and 6, and main text for details). The analysis was performed with R.
\subsection{Historical data}
\subsubsection*{Emergence date}

\begin{verbatim}
##              Df Sum Sq Mean Sq F value  Pr(>F)    
## Latitude          5   3374   674.8   17.89 3.2e-13 ***
## Residuals   120   4527    37.7                    
## 
## Signif. codes:  0 '***' 0.001 '**' 0.01 '*' 0.05 '.' 0.1 ' ' 1
\end{verbatim}

\begin{verbatim}
## 
##  Pairwise comparisons using t tests with pooled SD 
## 
##        44.5N  45.5N  46.5N  47.5N  48.5N
## 45.5N 0.20440   -       -       -       -      
## 46.5N 1.00000 0.64190   -       -       -      
## 47.5N 0.00574 1.00000 0.02647   -       -      
## 48.5N 2.0e-09 0.00021 2.0e-08 0.01493   -      
## 49.5N 1.5e-08 0.00102 1.4e-07 0.05283 1.00000
## 
## P value adjustment method: bonferroni
\end{verbatim}

\subsubsection*{Budburst date}

\begin{verbatim}
##              Df Sum Sq Mean Sq F value   Pr(>F)    
## Latitude          5   1332  266.43   21.72 1.94e-15 ***
## Residuals   120   1472   12.26                     
## 
## Signif. codes:  0 '***' 0.001 '**' 0.01 '*' 0.05 '.' 0.1 ' ' 1
\end{verbatim}

\begin{verbatim}
## 
##  Pairwise comparisons using t tests with pooled SD 
##  
##        44.5N  45.5N  46.5N  47.5N  48.5N 
## 45.5N 0.21247   -       -       -       -      
## 46.5N 1.00000 1.00000   -       -       -      
## 47.5N 0.00062 1.00000 0.01185   -       -      
## 48.5N 6.5e-09 0.00050 3.5e-07 0.18502   -      
## 49.5N 3.1e-12 8.8e-07 2.2e-10 0.00155 1.00000
## 
## P value adjustment method: bonferroni
\end{verbatim}

\subsubsection*{Mismatch}

\begin{verbatim}
##              Df Sum Sq Mean Sq F value  Pr(>F)    
## Latitude          5  545.7  109.13   11.08 8.7e-09 ***
## Residuals   120 1182.1    9.85                    
## 
## Signif. codes:  0 '***' 0.001 '**' 0.01 '*' 0.05 '.' 0.1 ' ' 1
\end{verbatim}

\begin{verbatim}
## 
##  Pairwise comparisons using t tests with pooled SD 
## 
##        44.5N  45.5N  46.5N  47.5N  48.5N
## 45.5N 0.5382    -      -       -      -     
## 46.5N 1.0000  0.5175   -       -      -     
## 47.5N 0.2684  1.0000 0.2572    -      -     
## 48.5N 1.4e-07 0.0014 1.3e-07 0.0038   -     
## 49.5N 0.0042  1.0000 0.0039  1.0000 0.2517
## 
## P value adjustment method: bonferroni
\end{verbatim}

\subsection{RCP 2.6 data}
\subsubsection*{Emergence date}

\begin{verbatim}
##               Df Sum Sq Mean Sq F value Pr(>F)    
## Latitude     5 305237   61047    3334 <2e-16 ***
## Residuals   7194 131707      18                   
## 
## Signif. codes:  0 '***' 0.001 '**' 0.01 '*' 0.05 '.' 0.1 ' ' 1
\end{verbatim}

\begin{verbatim}
## 
##  Pairwise comparisons using t tests with pooled SD 
##  
##        44.5N  45.5N  46.5N  47.5N  48.5N
## 45.5N <2e-16   -      -      -      -     
## 46.5N <2e-16 <2e-16   -      -      -     
## 47.5N <2e-16 <2e-16 <2e-16   -      -     
## 48.5N <2e-16 <2e-16 <2e-16 <2e-16   -     
## 49.5N <2e-16 <2e-16 <2e-16 <2e-16 <2e-16
## 
## P value adjustment method: bonferroni
\end{verbatim}

\subsubsection*{Budburst date}

\begin{verbatim}
##               Df Sum Sq Mean Sq F value Pr(>F)    
## Latitude     5  94260   18852    2896 <2e-16 ***
## Residuals   7194  46827       7                   
## 
## Signif. codes:  0 '***' 0.001 '**' 0.01 '*' 0.05 '.' 0.1 ' ' 1
\end{verbatim}

\begin{verbatim}
## 
##  Pairwise comparisons using t tests with pooled SD 
## 
##        44.5N  45.5N  46.5N  47.5N  48.5N
## 45.5N <2e-16   -      -      -      -    
## 46.5N <2e-16 <2e-16   -      -      -    
## 47.5N <2e-16 <2e-16 <2e-16   -      -    
## 48.5N <2e-16 <2e-16 <2e-16 <2e-16   -    
## 49.5N <2e-16 <2e-16 <2e-16 <2e-16 1    
## 
## P value adjustment method: bonferroni
\end{verbatim}

\subsubsection*{Mismatch}

\begin{verbatim}
##               Df Sum Sq Mean Sq F value Pr(>F)    
## Latitude     5  66174   13235    2316 <2e-16 ***
## Residuals   7194  41116       6                   
## 
## Signif. codes:  0 '***' 0.001 '**' 0.01 '*' 0.05 '.' 0.1 ' ' 1
\end{verbatim}

\begin{verbatim}
## 
##  Pairwise comparisons using t tests with pooled SD 
##  
##        44.5N  45.5N  46.5N  47.5N  48.5N
## 45.5N <2e-16   -      -      -      -     
## 46.5N <2e-16 <2e-16   -      -      -     
## 47.5N <2e-16 <2e-16 <2e-16   -      -     
## 48.5N <2e-16 <2e-16 <2e-16 <2e-16   -     
## 49.5N <2e-16 <2e-16 <2e-16 <2e-16 <2e-16
## 
## P value adjustment method: bonferroni
\end{verbatim}

\subsection{RCP 4.5 data}
\subsubsection*{Emergence date}

\begin{verbatim}
##               Df Sum Sq Mean Sq F value Pr(>F)    
## Latitude     5 139304   27861    1045 <2e-16 ***
## Residuals   7194 191768      27                   
## 
## Signif. codes:  0 '***' 0.001 '**' 0.01 '*' 0.05 '.' 0.1 ' ' 1
\end{verbatim}

\begin{verbatim}
## 
##  Pairwise comparisons using t tests with pooled SD 
##  
##        44.5N  45.5N  46.5N  47.5N  48.5N
## 45.5N <2e-16   -      -      -      -    
## 46.5N <2e-16 <2e-16   -      -      -    
## 47.5N <2e-16 1.000  <2e-16   -      -    
## 48.5N <2e-16 <2e-16 <2e-16 <2e-16   -    
## 49.5N <2e-16 <2e-16 <2e-16 <2e-16 0.006
## 
## P value adjustment method: bonferroni
\end{verbatim}

\subsubsection*{Budburst date}

\begin{verbatim}
##               Df Sum Sq Mean Sq F value Pr(>F)    
## Latitude     5  51477   10295    1046 <2e-16 ***
## Residuals   7194  70813      10                   
## 
## Signif. codes:  0 '***' 0.001 '**' 0.01 '*' 0.05 '.' 0.1 ' ' 1
\end{verbatim}

\begin{verbatim}
## 
##  Pairwise comparisons using t tests with pooled SD 
## 
##        44.5N  45.5N  46.5N  47.5N  48.5N 
## 45.5N <2e-16   -      -      -      -     
## 46.5N <2e-16 <2e-16   -      -      -     
## 47.5N <2e-16 0.62   <2e-16   -      -     
## 48.5N <2e-16 <2e-16 <2e-16 <2e-16   -     
## 49.5N <2e-16 <2e-16 <2e-16 <2e-16 <2e-16
## 
## P value adjustment method: bonferroni
\end{verbatim}

\subsubsection*{Mismatch}

\begin{verbatim}
##              Df Sum Sq Mean Sq F value Pr(>F)    
## Latitude     5  23457    4691   651.4 <2e-16 ***
## Residuals   7194  51815       7                   
## 
## Signif. codes:  0 '***' 0.001 '**' 0.01 '*' 0.05 '.' 0.1 ' ' 1
\end{verbatim}

\begin{verbatim}
## 
##  Pairwise comparisons using t tests with pooled SD 
## 
##        44.5N  45.5N  46.5N  47.5N  48.5N 
## 45.5N <2e-16   -      -      -      -     
## 46.5N <2e-16 <2e-16   -      -      -     
## 47.5N <2e-16 0.22   <2e-16   -      -     
## 48.5N <2e-16 <2e-16 <2e-16 <2e-16   -     
## 49.5N <2e-16 <2e-16 <2e-16 <2e-16 <2e-16
## 
## P value adjustment method: bonferroni
\end{verbatim}

\subsection{RCP 8.5 data}
\subsubsection*{Emergence date}


\begin{verbatim}
##               Df Sum Sq Mean Sq F value Pr(>F)    
## Latitude     5 155376   31075   727.7 <2e-16 ***
## Residuals   7194 307197      43                   
## 
## Signif. codes:  0 '***' 0.001 '**' 0.01 '*' 0.05 '.' 0.1 ' ' 1
\end{verbatim}

\begin{verbatim}
## 
##  Pairwise comparisons using t tests with pooled SD 
## 
##        44.5N  45.5N  46.5N  47.5N  48.5N
## 45.5N <2e-16   -      -      -      -    
## 46.5N <2e-16 <2e-16   -      -      -    
## 47.5N <2e-16 0.202  <2e-16   -      -    
## 48.5N <2e-16 <2e-16 <2e-16 <2e-16   -    
## 49.5N <2e-16 <2e-16 <2e-16 <2e-16 0.014
## 
## P value adjustment method: bonferroni
\end{verbatim}

\subsubsection*{Budburst date}

\begin{verbatim}
##               Df Sum Sq Mean Sq F value Pr(>F)    
## Latitude     5  58046   11609   690.5 <2e-16 ***
## Residuals   7194 120951      17                   
## 
## Signif. codes:  0 '***' 0.001 '**' 0.01 '*' 0.05 '.' 0.1 ' ' 1
\end{verbatim}

\begin{verbatim}
## 
##  Pairwise comparisons using t tests with pooled SD 
##  
##        44.5N  45.5N  46.5N  47.5N  48.5N
## 45.5N < 2e-16   -       -       -       -      
## 46.5N 8.2e-16 < 2e-16   -       -       -      
## 47.5N < 2e-16 0.0097  < 2e-16   -       -      
## 48.5N < 2e-16 < 2e-16 < 2e-16 < 2e-16   -      
## 49.5N < 2e-16 < 2e-16 < 2e-16 < 2e-16 < 2e-16
## 
## P value adjustment method: bonferroni
\end{verbatim}

\subsubsection*{Mismatch}

\begin{verbatim}
##               Df Sum Sq Mean Sq F value Pr(>F)    
## Latitude     5  25363    5073   572.3 <2e-16 ***
## Residuals   7194  63763       9                   
## 
## Signif. codes:  0 '***' 0.001 '**' 0.01 '*' 0.05 '.' 0.1 ' ' 1
\end{verbatim}

\begin{verbatim}
## 
##  Pairwise comparisons using t tests with pooled SD 
## 
##        44.5N  45.5N  46.5N  47.5N  48.5N  
## 45.5N < 2e-16   -       -       -       -      
## 46.5N < 2e-16 < 2e-16   -       -       -      
## 47.5N < 2e-16 1       < 2e-16   -       -      
## 48.5N < 2e-16 < 2e-16 < 2e-16 < 2e-16   -      
## 49.5N < 2e-16 < 2e-16 < 2e-16 < 2e-16 1.1e-11
## 
## P value adjustment method: bonferroni
\end{verbatim}

\bibliography{Supplements}
\end{document}